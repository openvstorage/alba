\documentclass{article}
\usepackage{hyperref}
\usepackage{listings}
\title{A Remote Key Value Store for Alba}
\begin{document}
\maketitle
\section{High Level Overview}
We need an alternative for \href{https://developers.seagate.com/display/KV/Kinetic+Open+Storage+Documentation+Wiki}{Seagate's Kinetic drives}.
Basically, we need a tcp/ip socket server in front of a local key value store.
But we also need some pieces of functionality that are currently absent
in these Kinetic drives.
The server manages a local database and accepts incoming connections from clients.
Clients can send requests that can be queries or updates.
For each client, the requests are handled in order,
and requests can be \emph{pipelined}: a client can send multiple request messages,
one after another before it starts reading responses.
There is no ordering between concurrent requests of different clients.
There are however some 'calls' to perform \emph{atomic multi-updates}.
You know that \emph{keys} are strings of maximum 4KB,
and \emph{values} are limited to 1MB,
just like with the original Kinetic drives.

\section{API}
The api is described from a client perspective.
\subsection{non-pipelined}
In the basic use case client opens a connection,
sends a message and reads a response, and repeats.
\begin{lstlisting}
  request = ...
  send(connection, request);
  response = receive(connection);
  # or shorter, which does the same as above
  #response = rpc(connection, request)
  # some way of getting the useful data out of the response
  ...
\end{lstlisting}
\subsection{pipelined}
\begin{lstlisting}
  request1 = ....
  send(connection,request1)
  request2 = ...
  send(connection,request2)
  ...
  response1 = receive(connection)
  response2 = receive(connection)
  ...
\end{lstlisting}
\subsection{Services}
\subsubsection{MultiGet}
  \begin{lstlisting}
    request = [key1,key2,key3] # list of strings
    response = rpc(connection, request)
  \end{lstlisting}

In the example, the response is a list of 3 string options:\texttt{[v1,v2,v3]}.
The value option \texttt{v1} is associated with \texttt{key1}.
It's actually a string pointer.
If there is no value for the corresponding key,
the value is a \texttt{NULL} pointer.
\subsubsection{Sequences}
  \begin{lstlisting}
    request = Sequence()
    request.add(Assert(key1, value))
    request.add(Assert(key2, None))
    request.add(Set(key3, value2))
    request.add(Set(key4, None))
    response = rpc(connection, request)
  \end{lstlisting}
  The sequence is applied within a transaction on server side.
  This means either all updates within the sequence succeed, or none at all.
  The asserts offer a simple way to bail out in case the situation in the database is not what was anticipated.
A \texttt{Set} to \texttt{None} means: delete the value if it existed.
Setting a non-existent key to None is a \emph{noop}.
\subsubsection{Range Queries}
  \begin{lstlisting}
    request = Range(first_key,first_included,
                    last_key, last_included, reverse, max_keys)
    response = rpc(connection, request)
  \end{lstlisting}
  The server side will return a list of keys, all within the range.
  If \texttt{first\_included} is \texttt{false} then the first key will not be
  in the response.
  The server can decide to return less than the \emph{max\_keys},
basically to pretect itself against superabundant queries,
but there needs to be a flag to find out the difference between
the following 2 cases:
\begin{itemize}
  \item{}you received all you asked for,
    but there are only this many keys in the range.
  \item{}you received less than you asked for,
    and if you need more, you'll have send a new query.
\end{itemize}
  If \texttt{reverse} is true,
  the keys are searched for and returned in a reverse order.
  For instance if the search is
  \begin{lstlisting}
  Range(first_key="j", last_key="k", reverse=true, max_keys=2)
  \end{lstlisting}
  and the keys "k0", "k1", "k2" exist,
  the system will return "k2" and "k1" in that order.

\subsubsection{Peer2Peer}

  \begin{lstlisting}
    request = Peer2PeerPush(target_ip, target_port,[key1,key2,key3])
    response = rpc(connection, request)
  \end{lstlisting}
Instructs the server to perform an atomic multi-set of key value pairs
to another server. It should fail if any one of these keys has no value.
\begin{lstlisting}
  request = Peer2PeerPushRange(target_ip, target_port,
                               first_key, first_key_included,
                               last_key  , last_key_included)
  response = rpc(connection, request)

\end{lstlisting}
  On the server, the range query is performed and the resulting
  list of key value pairs is sent to the target server as an atomic multi-set.
  The server could opt not to push all key value pairs in a range.
So the response needs to contain how many were pushed,
if they all were pushed, and if needed the continuation (first) key to push
the next part.
\section{Implementation Details}

\subsection{rpc protocol}
We need something that can be used from clients developed in multiple languages,
so therefore we need something that's fairly standard.
We want you to use \href{http://kentonv.github.io/capnproto/}{cap'n proto}.
\subsection{local key value store}
We don't want you to write a local key value store from scratch.
In fact \href{http://rocksdb.org/}{rocksdb} has everything you need.
In a second phase, we can look for doing something else (for example if \emph{rocksdb} can't handle a 4TB database).

\subsection{Also important}
Make sure you contact us before you start implementing something.
\end{document}
